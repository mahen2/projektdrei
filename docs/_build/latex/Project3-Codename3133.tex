% Generated by Sphinx.
\def\sphinxdocclass{report}
\documentclass[letterpaper,10pt,english]{sphinxmanual}
\usepackage[utf8]{inputenc}
\DeclareUnicodeCharacter{00A0}{\nobreakspace}
\usepackage{cmap}
\usepackage[T1]{fontenc}
\usepackage{babel}
\usepackage{times}
\usepackage[Bjarne]{fncychap}
\usepackage{longtable}
\usepackage{sphinx}
\usepackage{multirow}


\title{Project 3 - Codename 3133 Documentation}
\date{December 17, 2013}
\release{1.0}
\author{Maria Henkel, Maurice Schleussinger, Dennis Weber}
\newcommand{\sphinxlogo}{}
\renewcommand{\releasename}{Release}
\makeindex

\makeatletter
\def\PYG@reset{\let\PYG@it=\relax \let\PYG@bf=\relax%
    \let\PYG@ul=\relax \let\PYG@tc=\relax%
    \let\PYG@bc=\relax \let\PYG@ff=\relax}
\def\PYG@tok#1{\csname PYG@tok@#1\endcsname}
\def\PYG@toks#1+{\ifx\relax#1\empty\else%
    \PYG@tok{#1}\expandafter\PYG@toks\fi}
\def\PYG@do#1{\PYG@bc{\PYG@tc{\PYG@ul{%
    \PYG@it{\PYG@bf{\PYG@ff{#1}}}}}}}
\def\PYG#1#2{\PYG@reset\PYG@toks#1+\relax+\PYG@do{#2}}

\expandafter\def\csname PYG@tok@gd\endcsname{\def\PYG@tc##1{\textcolor[rgb]{0.63,0.00,0.00}{##1}}}
\expandafter\def\csname PYG@tok@gu\endcsname{\let\PYG@bf=\textbf\def\PYG@tc##1{\textcolor[rgb]{0.50,0.00,0.50}{##1}}}
\expandafter\def\csname PYG@tok@gt\endcsname{\def\PYG@tc##1{\textcolor[rgb]{0.00,0.27,0.87}{##1}}}
\expandafter\def\csname PYG@tok@gs\endcsname{\let\PYG@bf=\textbf}
\expandafter\def\csname PYG@tok@gr\endcsname{\def\PYG@tc##1{\textcolor[rgb]{1.00,0.00,0.00}{##1}}}
\expandafter\def\csname PYG@tok@cm\endcsname{\let\PYG@it=\textit\def\PYG@tc##1{\textcolor[rgb]{0.25,0.50,0.56}{##1}}}
\expandafter\def\csname PYG@tok@vg\endcsname{\def\PYG@tc##1{\textcolor[rgb]{0.73,0.38,0.84}{##1}}}
\expandafter\def\csname PYG@tok@m\endcsname{\def\PYG@tc##1{\textcolor[rgb]{0.13,0.50,0.31}{##1}}}
\expandafter\def\csname PYG@tok@mh\endcsname{\def\PYG@tc##1{\textcolor[rgb]{0.13,0.50,0.31}{##1}}}
\expandafter\def\csname PYG@tok@cs\endcsname{\def\PYG@tc##1{\textcolor[rgb]{0.25,0.50,0.56}{##1}}\def\PYG@bc##1{\setlength{\fboxsep}{0pt}\colorbox[rgb]{1.00,0.94,0.94}{\strut ##1}}}
\expandafter\def\csname PYG@tok@ge\endcsname{\let\PYG@it=\textit}
\expandafter\def\csname PYG@tok@vc\endcsname{\def\PYG@tc##1{\textcolor[rgb]{0.73,0.38,0.84}{##1}}}
\expandafter\def\csname PYG@tok@il\endcsname{\def\PYG@tc##1{\textcolor[rgb]{0.13,0.50,0.31}{##1}}}
\expandafter\def\csname PYG@tok@go\endcsname{\def\PYG@tc##1{\textcolor[rgb]{0.20,0.20,0.20}{##1}}}
\expandafter\def\csname PYG@tok@cp\endcsname{\def\PYG@tc##1{\textcolor[rgb]{0.00,0.44,0.13}{##1}}}
\expandafter\def\csname PYG@tok@gi\endcsname{\def\PYG@tc##1{\textcolor[rgb]{0.00,0.63,0.00}{##1}}}
\expandafter\def\csname PYG@tok@gh\endcsname{\let\PYG@bf=\textbf\def\PYG@tc##1{\textcolor[rgb]{0.00,0.00,0.50}{##1}}}
\expandafter\def\csname PYG@tok@ni\endcsname{\let\PYG@bf=\textbf\def\PYG@tc##1{\textcolor[rgb]{0.84,0.33,0.22}{##1}}}
\expandafter\def\csname PYG@tok@nl\endcsname{\let\PYG@bf=\textbf\def\PYG@tc##1{\textcolor[rgb]{0.00,0.13,0.44}{##1}}}
\expandafter\def\csname PYG@tok@nn\endcsname{\let\PYG@bf=\textbf\def\PYG@tc##1{\textcolor[rgb]{0.05,0.52,0.71}{##1}}}
\expandafter\def\csname PYG@tok@no\endcsname{\def\PYG@tc##1{\textcolor[rgb]{0.38,0.68,0.84}{##1}}}
\expandafter\def\csname PYG@tok@na\endcsname{\def\PYG@tc##1{\textcolor[rgb]{0.25,0.44,0.63}{##1}}}
\expandafter\def\csname PYG@tok@nb\endcsname{\def\PYG@tc##1{\textcolor[rgb]{0.00,0.44,0.13}{##1}}}
\expandafter\def\csname PYG@tok@nc\endcsname{\let\PYG@bf=\textbf\def\PYG@tc##1{\textcolor[rgb]{0.05,0.52,0.71}{##1}}}
\expandafter\def\csname PYG@tok@nd\endcsname{\let\PYG@bf=\textbf\def\PYG@tc##1{\textcolor[rgb]{0.33,0.33,0.33}{##1}}}
\expandafter\def\csname PYG@tok@ne\endcsname{\def\PYG@tc##1{\textcolor[rgb]{0.00,0.44,0.13}{##1}}}
\expandafter\def\csname PYG@tok@nf\endcsname{\def\PYG@tc##1{\textcolor[rgb]{0.02,0.16,0.49}{##1}}}
\expandafter\def\csname PYG@tok@si\endcsname{\let\PYG@it=\textit\def\PYG@tc##1{\textcolor[rgb]{0.44,0.63,0.82}{##1}}}
\expandafter\def\csname PYG@tok@s2\endcsname{\def\PYG@tc##1{\textcolor[rgb]{0.25,0.44,0.63}{##1}}}
\expandafter\def\csname PYG@tok@vi\endcsname{\def\PYG@tc##1{\textcolor[rgb]{0.73,0.38,0.84}{##1}}}
\expandafter\def\csname PYG@tok@nt\endcsname{\let\PYG@bf=\textbf\def\PYG@tc##1{\textcolor[rgb]{0.02,0.16,0.45}{##1}}}
\expandafter\def\csname PYG@tok@nv\endcsname{\def\PYG@tc##1{\textcolor[rgb]{0.73,0.38,0.84}{##1}}}
\expandafter\def\csname PYG@tok@s1\endcsname{\def\PYG@tc##1{\textcolor[rgb]{0.25,0.44,0.63}{##1}}}
\expandafter\def\csname PYG@tok@gp\endcsname{\let\PYG@bf=\textbf\def\PYG@tc##1{\textcolor[rgb]{0.78,0.36,0.04}{##1}}}
\expandafter\def\csname PYG@tok@sh\endcsname{\def\PYG@tc##1{\textcolor[rgb]{0.25,0.44,0.63}{##1}}}
\expandafter\def\csname PYG@tok@ow\endcsname{\let\PYG@bf=\textbf\def\PYG@tc##1{\textcolor[rgb]{0.00,0.44,0.13}{##1}}}
\expandafter\def\csname PYG@tok@sx\endcsname{\def\PYG@tc##1{\textcolor[rgb]{0.78,0.36,0.04}{##1}}}
\expandafter\def\csname PYG@tok@bp\endcsname{\def\PYG@tc##1{\textcolor[rgb]{0.00,0.44,0.13}{##1}}}
\expandafter\def\csname PYG@tok@c1\endcsname{\let\PYG@it=\textit\def\PYG@tc##1{\textcolor[rgb]{0.25,0.50,0.56}{##1}}}
\expandafter\def\csname PYG@tok@kc\endcsname{\let\PYG@bf=\textbf\def\PYG@tc##1{\textcolor[rgb]{0.00,0.44,0.13}{##1}}}
\expandafter\def\csname PYG@tok@c\endcsname{\let\PYG@it=\textit\def\PYG@tc##1{\textcolor[rgb]{0.25,0.50,0.56}{##1}}}
\expandafter\def\csname PYG@tok@mf\endcsname{\def\PYG@tc##1{\textcolor[rgb]{0.13,0.50,0.31}{##1}}}
\expandafter\def\csname PYG@tok@err\endcsname{\def\PYG@bc##1{\setlength{\fboxsep}{0pt}\fcolorbox[rgb]{1.00,0.00,0.00}{1,1,1}{\strut ##1}}}
\expandafter\def\csname PYG@tok@kd\endcsname{\let\PYG@bf=\textbf\def\PYG@tc##1{\textcolor[rgb]{0.00,0.44,0.13}{##1}}}
\expandafter\def\csname PYG@tok@ss\endcsname{\def\PYG@tc##1{\textcolor[rgb]{0.32,0.47,0.09}{##1}}}
\expandafter\def\csname PYG@tok@sr\endcsname{\def\PYG@tc##1{\textcolor[rgb]{0.14,0.33,0.53}{##1}}}
\expandafter\def\csname PYG@tok@mo\endcsname{\def\PYG@tc##1{\textcolor[rgb]{0.13,0.50,0.31}{##1}}}
\expandafter\def\csname PYG@tok@mi\endcsname{\def\PYG@tc##1{\textcolor[rgb]{0.13,0.50,0.31}{##1}}}
\expandafter\def\csname PYG@tok@kn\endcsname{\let\PYG@bf=\textbf\def\PYG@tc##1{\textcolor[rgb]{0.00,0.44,0.13}{##1}}}
\expandafter\def\csname PYG@tok@o\endcsname{\def\PYG@tc##1{\textcolor[rgb]{0.40,0.40,0.40}{##1}}}
\expandafter\def\csname PYG@tok@kr\endcsname{\let\PYG@bf=\textbf\def\PYG@tc##1{\textcolor[rgb]{0.00,0.44,0.13}{##1}}}
\expandafter\def\csname PYG@tok@s\endcsname{\def\PYG@tc##1{\textcolor[rgb]{0.25,0.44,0.63}{##1}}}
\expandafter\def\csname PYG@tok@kp\endcsname{\def\PYG@tc##1{\textcolor[rgb]{0.00,0.44,0.13}{##1}}}
\expandafter\def\csname PYG@tok@w\endcsname{\def\PYG@tc##1{\textcolor[rgb]{0.73,0.73,0.73}{##1}}}
\expandafter\def\csname PYG@tok@kt\endcsname{\def\PYG@tc##1{\textcolor[rgb]{0.56,0.13,0.00}{##1}}}
\expandafter\def\csname PYG@tok@sc\endcsname{\def\PYG@tc##1{\textcolor[rgb]{0.25,0.44,0.63}{##1}}}
\expandafter\def\csname PYG@tok@sb\endcsname{\def\PYG@tc##1{\textcolor[rgb]{0.25,0.44,0.63}{##1}}}
\expandafter\def\csname PYG@tok@k\endcsname{\let\PYG@bf=\textbf\def\PYG@tc##1{\textcolor[rgb]{0.00,0.44,0.13}{##1}}}
\expandafter\def\csname PYG@tok@se\endcsname{\let\PYG@bf=\textbf\def\PYG@tc##1{\textcolor[rgb]{0.25,0.44,0.63}{##1}}}
\expandafter\def\csname PYG@tok@sd\endcsname{\let\PYG@it=\textit\def\PYG@tc##1{\textcolor[rgb]{0.25,0.44,0.63}{##1}}}

\def\PYGZbs{\char`\\}
\def\PYGZus{\char`\_}
\def\PYGZob{\char`\{}
\def\PYGZcb{\char`\}}
\def\PYGZca{\char`\^}
\def\PYGZam{\char`\&}
\def\PYGZlt{\char`\<}
\def\PYGZgt{\char`\>}
\def\PYGZsh{\char`\#}
\def\PYGZpc{\char`\%}
\def\PYGZdl{\char`\$}
\def\PYGZhy{\char`\-}
\def\PYGZsq{\char`\'}
\def\PYGZdq{\char`\"}
\def\PYGZti{\char`\~}
% for compatibility with earlier versions
\def\PYGZat{@}
\def\PYGZlb{[}
\def\PYGZrb{]}
\makeatother

\begin{document}

\maketitle
\tableofcontents
\phantomsection\label{index::doc}


Contents:


\chapter{\texttt{project\_3} --- Informetrische Untersuchung mit Mendeley und Matplotlib}
\label{project_3:module-project_3}\label{project_3::doc}\label{project_3:project-3-informetrische-untersuchung-mit-mendeley-und-matplotlib}\label{project_3:welcome-to-project-3-codename-3133-s-documentation}\index{project\_3 (module)}

\section{Einleitung}
\label{project_3:einleitung}
Mit dem vorliegenden Programm werden Publikationen informetrisch untersucht. Als Basis für den Datensatz dient das Literaturverwaltungsprogramm Mendeley. Mendeley beinhaltet nicht nur eine Desktopversion zum Verwalten von Referenzen und PDF-Dateien, sondern auch ein soziales Online-Netzwerk für den Austausch und Kollaborationen zwischen Forschern. Das Programm greift auf die Publikationsdaten über die Mendeley-API zu, wertet sie aus und visualisiert sie mit Hilfe von Matplotlib.


\section{Funktionsumfang}
\label{project_3:funktionsumfang}
Folgende Daten von den Mendeley-Servern gesammelt und visualisiert:
\begin{itemize}
\item {} 
Verteilung der Publikationen auf die letzten 10 Jahre

\item {} 
Top 20 Tags in der Kategorie „Computer and Information Science“

\item {} 
Top 10 Publikationen aus der „Nature“

\item {} 
Special: Publikationen von Prof. Wolfang G. Stock
\begin{itemize}
\item {} 
Publikationsanzahl pro Jahr

\item {} 
Co-Autoren Ranking

\end{itemize}

\item {} 
Häufigkeit des Tags „ontology“ in allen Kategorien im Jahr 2011

\end{itemize}


\section{Benutzung}
\label{project_3:benutzung}
Wird das Programm direkt ausgeführt werden zunächst alle Daten über die Mendeley-API gesammelt. Dabei wird davon ausgegangen, dass sich eine Datei ``config.json'' im gleichen Verzeichnis befindet. Hierfür kann ``bsp config.json'' entsprechend angepasst und umbenannt werden.

Unter Umständen kann die Datensammlung das Limit der Mendeley-API ausreizen. Ist dies der Fall müssen zunächst einzelne Abschnitte auskommentiert werden. Die Datensammlung wird im Code durch den Kommentar \emph{\# \#\#\# Collect the required data \#\#\#} eingeleitet.

Die gesammelten Daten werden autmatisch zwischengespeichert, sodass nicht bei jedem Programmaufruf auf die API zugegriffen werden muss. Um die zwischengespeicherten Daten zu verwenden, kann die gesammte Datensammlung auskommentiert werden (markiert durch * \# \#\#\# Collect the required data \#\#\#*).
Das Programm wird bereits mit Daten vom 17 Dezember 2013 ausgeliefert. Die Rechte dieser Daten liegt bei Mendeley.


\section{Auswertung und Interpretation}
\label{project_3:auswertung-und-interpretation}
Blablub informetrie


\section{Funktionen}
\label{project_3:funktionen}\index{save\_as\_pickle() (in module project\_3)}

\begin{fulllineitems}
\phantomsection\label{project_3:project_3.save_as_pickle}\pysiglinewithargsret{\code{project\_3.}\bfcode{save\_as\_pickle}}{\emph{p\_object}, \emph{filename}}{}
Speichert einen beliebigen Datentyp \emph{p\_object} als eine Pickle-Datei mit dem Dateinamen \emph{filename}.py. \emph{filename} sollte als String übergeben werden.

\end{fulllineitems}

\index{open\_from\_pickle() (in module project\_3)}

\begin{fulllineitems}
\phantomsection\label{project_3:project_3.open_from_pickle}\pysiglinewithargsret{\code{project\_3.}\bfcode{open\_from\_pickle}}{\emph{filename}}{}
Öffnet eine zuvor mit {\hyperref[project_3:project_3.save_as_pickle]{\code{save\_as\_pickle()}}} erstellte Pickle-Datei, bzw. eine beliebe Pickle-Datei, die genau einen (verschachtelten) Datentyp enthält. Der Name der Datei wird durch \emph{filename} bestimmt.

\end{fulllineitems}

\index{draw\_barchart() (in module project\_3)}

\begin{fulllineitems}
\phantomsection\label{project_3:project_3.draw_barchart}\pysiglinewithargsret{\code{project\_3.}\bfcode{draw\_barchart}}{\emph{names}, \emph{values}, \emph{ylabel}, \emph{title}}{}
Zeichnet ein Balkendiagramm für einen bestimmten Datensatz (definiert über die Parameter).
Eine Liste von Strings als Parameter \emph{names} bestimmt die Labels der x-Achse, \emph{values} eine Anzahl von Daten in Form einer Liste. Die Beschriftung der y-Achse wird durch einen String \emph{ylabel} bestimmt. Das Parameter \emph{title} gibt den Titel des Diagramms in Form eines Strings an.

\end{fulllineitems}

\index{draw\_piechart() (in module project\_3)}

\begin{fulllineitems}
\phantomsection\label{project_3:project_3.draw_piechart}\pysiglinewithargsret{\code{project\_3.}\bfcode{draw\_piechart}}{\emph{names}, \emph{values}}{}
Zeichnet ein Kreisdiagramm für einen bestimmten Datensatz (definiert über die Parameter). Eine Liste von Strings als Parameter \emph{names}, bestimmt die Labels jedes Teilstücks des Kreisdiagramms, \emph{values} ist eine Liste von Daten in Form von Integern.

\end{fulllineitems}

\index{draw\_timeline() (in module project\_3)}

\begin{fulllineitems}
\phantomsection\label{project_3:project_3.draw_timeline}\pysiglinewithargsret{\code{project\_3.}\bfcode{draw\_timeline}}{\emph{names}, \emph{values}, \emph{ylabel}, \emph{title}}{}
Zeichnet eine Timeline für einen bestimmten Datensatz (definiert über die Parameter).
Eine Liste von Strings als Parameter \emph{names} bestimmt die Labels der x-Achse, \emph{values} eine Anzahl von Daten in Form einer Liste. Die Beschriftung der y-Achse wird durch einen String \emph{ylabel} bestimmt. Das Parameter \emph{title} gibt den Titel des Diagramms in Form eines Strings an.

\end{fulllineitems}



\section{Referenzen}
\label{project_3:referenzen}
Folgende Module fanden neben der Python Standard Library Verwendung:
\begin{itemize}
\item {} 
mendeley-oapi-example-master \href{https://github.com/Mendeley/mendeley-oapi-example/}{https://github.com/Mendeley/mendeley-oapi-example:}

\item {} 
Matplotlib: \href{http://matplotlib.org/}{http://matplotlib.org}

\end{itemize}

Alle Daten wurden mithilfe von Mendeley gesammelt:
\begin{itemize}
\item {} 
Mendeley: \href{http://mendeley.com/}{http://mendeley.com}

\item {} 
Mendeley-API: \href{http://dev.mendeley.com/}{http://dev.mendeley.com}

\end{itemize}


\chapter{Indices and tables}
\label{index:indices-and-tables}\begin{itemize}
\item {} 
\emph{genindex}

\item {} 
\emph{modindex}

\item {} 
\emph{search}

\end{itemize}


\renewcommand{\indexname}{Python Module Index}
\begin{theindex}
\def\bigletter#1{{\Large\sffamily#1}\nopagebreak\vspace{1mm}}
\bigletter{p}
\item {\texttt{project\_3}}, \pageref{project_3:module-project_3}
\end{theindex}

\renewcommand{\indexname}{Index}
\printindex
\end{document}
